\documentclass{article}
\usepackage{cite}
\usepackage{graphicx}
\usepackage{multirow}
\usepackage[table,xcdraw]{xcolor}
\usepackage{hyperref}

\newcommand\coordinator[1]{\begin{flushleft}\small\textit{#1}\end{flushleft}}

\title{Proceedings of the OHBM Hackathon 2023}
\author{Yu-Fang Yang, Anibal Sólon Heinsfeld}
\date{December 2023}

\begin{document}

\maketitle

\section{Introduction}

The OHBM Brainhack is a satellite hackathon of the Organization for Human Brain Mapping (OHBM) annual meeting.
It serves as a hub for neuroscientists of various disciplines and career stages to collaborate in a dynamic, open, and inclusive environment.
Through its unique format, the Brainhack promotes open science by encouraging participants to engage in collaborative project development, skill-sharing, and community building.
These efforts are deeply aligned with the mission of OHBM’s Open Science Special Interest Group (OSSIG), fostering transparency, reproducibility, and accessibility in neuroscience research.

The 2023 OHBM Brainhack centered around three core pillars:
1) \emph{Inclusivity} – Cultivating an environment where participants from all backgrounds, disciplines, and expertise levels feel welcome and empowered to contribute meaningfully;
2) \emph{Empowerment} – Providing hands-on training and skill-building opportunities through structured and unstructured learning sessions to enhance participants' research abilities;
3) \emph{Innovation} – Encouraging the development of novel and interdisciplinary projects that push the boundaries of neuroimaging and neuroscience research.
This year’s event continued to build upon the foundations laid by previous Brainhack editions.
The hybrid format introduced in earlier years was retained to reduce the carbon footprint \cite{Epp2023} and address barriers such as lack of funding or visa challenges.
This hybrid approach allowed broader participation from the global neuroscience community and reinforced the organizers' commitment to diversity, inclusion, and equitable access.

Notable new initiatives in 2023 included the Mini-Grant Initiative, designed to support projects prioritizing open science, diversity, and interdisciplinary collaboration.
This initiative, while well-received, revealed mixed responses regarding its competitive nature, highlighting important insights for future event planning.
This system fostered meaningful connections and enhanced participants' engagement with the Brainhack community.

In addition to these new elements, the Train-Track and Hack-Track components remained central to the event's structure.
The Train-Track offered structured tutorials and self-directed learning sessions on topics such as coding, machine learning, and reproducible research practices.
Meanwhile, the Hack-Track provided an opportunity for participants to collaborate on open projects submitted by community members, with many focusing on developing open-source software tools and pipelines for neuroimaging analysis.

\begin{figure}[h]
    \centering
    \includegraphics[width=0.2\textwidth]{images/placeholder.png}
    \label{fig:demographics}
    \caption{
        The four panels illustrate key demographic and participation trends for the OHBM BrainHack 2023 participants.
        \textbf{Panel A} shows the geographic distribution, with Canada, the United States, and France as the most represented countries among 13 in total.
        \textbf{Panel B} presents the career status distribution, with Early Career Researchers (ECRs)—including PhD students and postdoctoral researchers—making up half of the attendees, followed by established faculty (31.2\%) and Master's students (12.5\%).
        \textbf{Panel C} depicts participant satisfaction ratings for sessions such as Unconference, TrainTrack, and HackTrack, showing higher satisfaction for HackTrack sessions overall.
        \textbf{Panel D} examines format preferences (hybrid vs. onsite), with a stronger preference for hybrid participation among ECRs and attendees with more prior experience at BrainHack events, while onsite preference was higher among established faculty.
    }
\end{figure}

\section{Train-Track}
\coordinator{Andrea Gondová}

The Train-Track was an educational component aiming to create a supportive environment for knowledge exchange, allowing participants toenhance their technical skills and knowledge through both structured and flexible learning sessions. Suggested training materials covered diverse range of topics, including coding skills, good coding and best research practices, and machine learning. These materials consisted of tutorials and/or pre-recorded videos, largely sourced from the BrainHack School (for more information, see Table \ref{tab:traintrack-material} for training materials and links). Additionally, participants were encouraged to customize their learning experience  by self-organizing  study group with others sharing similar learning goals. Onsite sign-ups were also avaliable for mentor-mentee pairings, offering antenddes opportunities to gain expertise in advanced topics while expanding their professional networking. Various communication platforms, including in-person sessions and Discord, were used to facilitate collaborative learning.

Among the 31 respondents to the post-hackathon survey, 23\% participated in Train-Track sessions, with topics ranging from coding and machine learning to version control and coding best practices. “Good practices” was the most recommended topic by participants. However, feedback suggested a need for more structure and planning, such as scheduled sessions and topic prompts, to improve accessibility and participation. Organizers acknowledged the value of increased structure while maintaining Train-Track’s dynamic and community-driven nature. The goal remains to provide a flexible learning environment with suggested materials for inspiration, rather than prescriptive guidelines. To improve future iterations of Train-Track, organizers plan to introduce the program more clearly at the start of the event, encourage early interest in specific topics, and refine the program’s structure and objectives to better support participants.

\begin{table}[]
    \begin{tabular}{|l|l|c|}
    \hline
    \multicolumn{1}{|c|}{\textbf{Topic}}        & \multicolumn{1}{|c|}{\textbf{Session}}         & \multicolumn{1}{|c|}{\textbf{Link}}                                                                                                 \\ \hline
    \textbf{Getting started}                    & Setting-up a hack friendly environment         & \href{https://psy6983.brainhackmtl.org/modules/installation/}{Link}                                                      \\ \hline
                                                & Intro to jupyter notebooks                     & \href{https://docs.jupyter.org/en/latest/start/index.html}{Link}                                                         \\ \cline{2-3} 
                                                & Intro to bash                                  & \href{https://psy6983.brainhackmtl.org/modules/introduction\_to\_terminal/}{Link}                                        \\ \cline{2-3} 
                                                & Python: Data analysis                          & \href{https://psy6983.brainhackmtl.org/modules/python\_data\_analysis/}{Link}                                            \\ \cline{2-3} 
                                                & Python: Visualisation                          & \href{https://psy6983.brainhackmtl.org/modules/python\_visualization/}{Link}                                             \\ \cline{2-3} 
                                                & Python: Writing a script                       & \href{https://psy6983.brainhackmtl.org/modules/python\_scripts/}{Link}                                                   \\ \cline{2-3} 
    \multirow{-6}{*}{\textbf{Coding skills}}    & Python: Packaging (pipy)                       & \href{https://psy6983.brainhackmtl.org/modules/packaging/}{Link}                                                         \\ \hline
                                                & Using Git and Github                           & \href{https://youtu.be/zh\_WFv0uk7w}{Link} \href{https://psy6983.brainhackmtl.org/modules/git\_github/}{Link} \\ \cline{2-3} 
                                                & Data management with Datalad                   & \href{https://psy6983.brainhackmtl.org/modules/datalad/}{Link} \href{https://youtu.be/QsAqnP7TwyY}{Link}      \\ \cline{2-3} 
    \multirow{-5}{*}{\textbf{Version control}}  & Containers for science                         & \href{https://youtu.be/pc3YOZUG3lQ}{Link}                                                                                \\ \hline
                                                & How to write good code                         & \href{https://youtu.be/gfPP2pQ8Rms}{Link}                                                                                \\ \cline{2-3} 
                                                & Reproducible workflows: The Turing Way         & \href{https://youtu.be/tk2eZSrM8oA}{Link}                                                                                \\ \cline{2-3} 
    \multirow{-3}{*}{\textbf{Good practices}}   & Reproducible workflows: Transparent MRI        & \href{https://youtu.be/dSOQgyuL51U}{Link}                                                                                \\ \hline
                                                & Machine Learning basics                        & \href{https://psy6983.brainhackmtl.org/modules/machine\_learning\_basics/}{Link}                                         \\ \cline{2-3} 
                                                & Machine Learning for neuroimaging              & \href{https://psy6983.brainhackmtl.org/modules/machine\_learning\_neuroimaging/}{Link}                                   \\ \cline{2-3} 
    \multirow{-3}{*}{\textbf{Machine learning}} & Machine Learning Reproducibility Checklist     & \href{https://www.cs.mcgill.ca/$\sim$jpineau/ReproducibilityChecklist.pdf}{Link}                                         \\ \hline
    \end{tabular}
    \caption{Training materials provided for the Train-Track sessions at OHBM BrainHack 2023, including tutorials, pre-recorded videos, and resource links for coding, machine learning, and best practices.}
    \label{tab:traintrack-material}
\end{table}

\section{Buddy System}
\coordinator{Sina Mansour L.}

The Buddy System was established to support and engage newcomers at the Hackathon.
This initiative pairs participants in a mentor-mentee fashion, with mentors guiding interests, projects to join, and networking opportunities throughout the Hackathon.
Introduced in the previous 2022 edition, the system was refined for 2023 to create a more dynamic and natural community structure (see Figure \ref{fig:buddy-system}).
Pairs were given the freedom to choose their buddies to avoid any biases in matching, ensuring a fair and supportive environment for all participants.
In doing so, first-time attendees and more experienced brainhackers were encouraged to network and get to know each other while also sharing information about what to expect from the brainhack.
This system has been instrumental in helping first-time attendees feel welcomed and integrated into the community.

\begin{figure}[h]
    \centering
    \includegraphics[width=0.2\textwidth]{images/placeholder.png}
    \label{fig:buddy-system}
    \caption{
        (a) Front, and (b) back of a buddy card.
        Buddy cards were distributed amongst all attendees who expressed interest in participating in the refined dynamic buddy system.
        Individualized who participated were asked to share their cards with other brainhackers during lunch and coffee breaks.
        The cards acted as catalist to start a conversation and exchange thoughts, experiences, and contact information.
    }
\end{figure}

\section{Hack-Track}
\coordinator{Qing (Vincent) Wang}

The Hack-Track is designed for hands-on project development, allowing participants to collaborate on diverse and innovative projects.
Before the OHBM Hackathon 2023 began, project leaders submitted their proposals to our repository on GitHub, setting the stage for a dynamic and inclusive event.
A pre-Hackathon online session advertised the schedule and introduced participants to the online platform.
At the start of the Brainhack, project leaders pitched their projects, outlining goals, current status, and future aspirations.

The projects covered a wide range of topics, including coding, research tool development, documentation, community guideline development, and visualization techniques.
This variety allowed participants to choose projects based on their interests and goals.
Some focused deeply on a single project, while others divided their time among multiple projects for a broader experience.
Participants selected projects to contribute to their expertise or to acquire new skills and knowledge.
Hackathon projects often introduced novel methods or programming languages (e.g., Julia, WebAssembly, new visualizations), providing participants with practical skills for future endeavors.
Beyond the technical and educational aspects, the event also fostered community building, with many participants drawn to projects for the opportunity to collaborate and learn from new peers.
Over three intensive days, 33 projects advanced significantly, thanks to the collaborative efforts of 167 scientists from across the globe.
Despite the challenges of a hybrid event, especially in synchronizing video streams, we utilized various technologies within Discord to facilitate seamless participant exchanges.
This approach ensured that all participants, regardless of geographical location, could fully engage in the hackathon's collaborative spirit.

\section{Public Relations and Outreach}
\coordinator{Bruno Hebling Vieira, Xinhui Li}

Following the successful engagement from the OHBM Brainhack 2022, prior to the Hackathon, we leveraged various social media platforms, including Twitter, Mattermost, and WeChat, to announce registration information, and promote community engagement by sharing live project presentations during the event.
Throughout the Hackathon, Discord served as the primary communication channel between organisers and participants as the general channel (Brainhack announcement) has been created for Brainhack 2022 and many of the existing projects can continue the channels for 2023 without the need to recreate the channel.
A general channel was used to announce major events, while dedicated channels were automatically generated for each proposed project, facilitating focused group discussion among interested participants regarding project specifics.

Overall, digital outreach remains a fundamental tool for dissemination, with approximately 30\% of the pre-hackathon survey respondents reporting that they learned about the event from social media or the website.

\section{Projects}

\subsection{BrainViewer: Seamless Brain Mesh Visualization with WebGL}
Authors: Florian Rupprecht, Reinder Vos de Wael

BrainViewer, a novel JavaScript package, is poised to transform how researchers and developers interact with brain meshes by enabling seamless visualization directly within web browsers. By harnessing the power of ThreeJS, BrainViewer offers a responsive and adaptable viewing experience. Its integration with the JavaScript event system facilitates the creation of intricate and customizable behaviors, enhancing its utility.

BrainViewer is currently in an active phase of development. With our commitment to meeting the demands of our short-term projects, we anticipate the first release of BrainViewer within the coming months. BrainViewer is available at https://github.com/cmi-dair/brainviewer. Upon release, it will be installable through the npm package management system. A live demo that moves based on microphone input is available at https://cmi-dair.github.io/brainviewer-demo/.

BrainViewer harnesses the power of WebGL and WebGPU (depending on device capabilities) to efficiently render brain meshes within web browsers, delivering a visually engaging experience. Test users have consistently praised the smoothness of interaction. It supports a wide range of devices, from large PC screens with mouse and keyboard to compact mobile screens with touch interfaces, ensuring seamless deployment across ecosystems. Moreover, BrainViewer integrates with the JavaScript event system, enabling users to define complex and tailored behaviors that enhance interactivity and exploration.

\subsection{Clinica's image processing pipeline QC}
Authors: Matthieu Joulot, Ju-Chi Yu

The goal of the project was to evaluate different quality check (QC) metrics and visuals for pipelines currently existing in Clinica \cite{routier2021clinica}, which deal with registration or segmentation. That way, we would be able to find a few good metrics to separate the good images from the bad or moderately good ones, which the users would check using some visual we would generate for them.

We were able to output seven available QC metrics for registration, and with PCA, we evaluated their factor structure and kept only two metrics that captures the most QC information: 1) the correlation coefficients between the MNI template (target) and the subject’s brain image, and 2) the dice score (i.e., a type of distance measure) obtained by a brain extraction tool (BET) algorithm called HD-BET.

We will look further into our pipeline based on this result, so that we can have a better idea of what we want to implement in the future version of Clinica, which will include QC.

\begin{figure}[h]
    \centering
    \includegraphics[width=0.8\textwidth]{images/clinica-1.png}
    \label{fig:clinica-1}
    \caption{
        Scatterplot of HD BET Dice probability and correlation ratio between reference and moving images. Red color means both metrics are below the threshold, blue or purple means one of the metrics is below the threshold, green means both metrics are above the threshold.
    }
\end{figure}

\section{Mini-Grant Initiative}

This year, we introduced Mini-Grant initiative, generously supported by sponsors, aimed at empowering and supporting Brainhackers. The grants were distributed across nine categories:
\begin{itemize}
\item Open Science Mini-Grant: Funding projects promoting open science principles like data sharing and open-source software.
\item Diversity and Inclusion Mini-Grant: Supporting initiatives that enhance diversity and inclusion in science or academia.
\item Public Science Communication Mini-Grant: Backing efforts to communicate scientific research to the public in an accessible manner.
\item Innovative Teaching Method Mini-Grant: Encouraging creative approaches to science education.
\item Interdisciplinary Collaboration Mini-Grant: Fostering cross-disciplinary research to address complex challenges.
\item Student-Led Research Mini-Grant: Supporting science projects led by students.
\item Global Science Collaboration Mini-Grant: Encouraging international research collaborations.
\item Early Career Researcher Mini-Grant: Funding early-career scientists to conduct independent projects.
\item Science and Art Integration Mini-Grant: Funding projects that blend scientific concepts with artistic expression.
\end{itemize}

These grants were based on project merit, with nominations made by teams or peers after the project pitches and rhyming battle.
However, the response was less enthusiastic than anticipated, receiving only a limited number of nominations.
This outcome prompted a reflection on the initiative's alignment with the community's core values.
The competitive nature of the grant-awarding process might resonate poorly with the collaborative and inclusive ethos of the Brainhack community.
In light of these reflections, a decision was made to retain the funds for future Brainhack events, with a new approach in mind.
For the next OHBM Hackathon editions, the organizers could initiate the funding process earlier, allowing participants to submit proposals during registration.
This process will enable us to review the needs of the projects in advance and provide support where it's most needed, such as for essential equipment like webcams or help with travel arrangement costs.
The pivotal strategy aims to demonstrate our commitment to the community and ensure that support is available from the outset, fostering a more inclusive and resourceful environment

\section{Rhyming Battle}

One of the most memorable and engaging events at the OHBM Brainhack 2023 was the 'Rhyming Battle.'
Held before the project presentations, this unique event featured three participants who artistically expressed their experiences and challenges encountered during the hackathon.
They crafted short poems about their projects and the various aspects they faced during Brainhack, bringing a touch of lighthearted humour to their scientific endeavours \textcolor{red}{(see Appendix)}.
The Rhyming Battle provided an entertaining break from the Hackathon's more scientific aspects.
It allowed attendees to resonate with each other's experiences, celebrating the struggles and triumphs inherent in scientific projects.
The event showcased the creative and vibrant spirit of the Brainhack community.
Voting for the winner was not straightforward, as the attendees were equally good.
The prize was a single 3D homemade sculpture representing the Brainhack 2023 logo, made for the winner of the battle \textcolor{red}{(Figure x)}.
However, the enthusiasm of the battlers and the public was overwhelming, and we decided to 3D print the same sculpture for every battle participant.
Ultimately, the spirit of Brainhack community prevailed \cite{gau2021brainhack} \textcolor{red}{(Gau et al., 2021; Levitis et al., 2021)}, and every participant was celebrated as a winner.

\section{Conclusion}

The OHBM Brainhack 2023 exemplified a vibrant fusion of education, innovation, and community spirit.
Through initiatives like Train-Track, Hack-Track, the Buddy System, the Rhyming Battle, and the Mini-Grant program, the event fostered an environment rich in learning, creativity, and collaboration.
These elements, coupled with a commitment to open science and diversity, underscored the gathering's success in nurturing a dynamic scientific community.
Reflecting on the experiences and feedback from this year, Brainhack is poised to evolve, embracing lessons learned to enhance future hackathons.
This continuous improvement and deepened community engagement are set to further empower and inspire participants in the years to come.

\bibliography{refs}{}
\bibliographystyle{plain}

\end{document}
